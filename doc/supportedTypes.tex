\section{Supported Types}

Here follows a list of ETN currently supported types and how they are encoded.

\begin{itemize}

	\item Bool: represents the truth values true and false. Encoded within a byte: 1 for true, 0 for false. Represented in Go by type "bool".
	
	\item Integer: represents the set of n-bit signed/unsigned integers. Encoded in n-bit little-endian. An n-bit type is always encoded using n bits. Represented in Go by types "uint8", "uint16", "uint32", "uint64", "int8", "int16", "int32", "int64".
	
	\item Floating Point: represents the set of n-bit floating point numbers. Encoded in little-endian IEE-754. Represented in Go by types "float32", "float64".
	
	\item String: represents the set of Unicode strings. Encoded using UTF-8. The length in bytes "l" of the UTF-8 encoded string is sent first as a uint32, followed by the l bytes themselves. Represented in Go by type "string".
	
	\item Tuple: an ordered list of elements of a given type. The length in elements l of a tuple value is sent as a uint32, followed by l encoded values of the tuple's element type. Represented in Go by "array" and "slice" types.
	
	\item Dictionary: a map of keys to values. A map is an unordered group of elements of one type, called the element type, indexed by a set of unique keys of another type, called the key type. The number of map elements is its length l, which is encoded first as a uint32 followed by l pairs (key, value) encoded data.
	
	\item Pointer: a reference to a value of a given type. The way a pointer value is classified and encoded depends on whether or not the pointer is nil and whether it has already been encoded as part of a larger value (e.g. as part of a circular value).
	If a pointer value is nil, PNIL is encoded within a byte and encoding is complete.
	If a pointer value has been encoded before as part of the same, larger value, PIDX is encoded within a byte, followed by the index of the
previously encoded value.
Otherwise, PVAL is encoded within a byte, followed by the encoded value that the pointer references.
Represented in Go by "pointer" types.

	\item Struct: a composite type containing named fields with specified types. Names are ignored when sending or receiving structs. A struct is sent by encoding each field in the order specified by the type of the structure. Represented in Go by struct types. In order to encode or decode a struct, all fields must be exported.
	
	\item Any: a reference to a value of any type. Sent by encoding the hash for the type of the concrete value references as a Tuple, followed by the encoded value that the any references. Represented in Go by "empty interface" types. In order to encode or decode a value contained in an interface, the hash for the value's type must have been registered in the typetable (se tt.go for more details).

\end{itemize}
