\subsection{trie.go} 

\begin{itemize}

\item \emph{type trie struct}\\
Type trie is used to build a tree. It contains a byte (used as index) and interface and two pointers to trie type (one to the child and one to the sibling)

\item \emph{func (t *trie) prefix(b []byte) (s []byte, r *trie)}\\
Function of trie type. It finds the longest present prefix of b. It returns the prefix and pointer to a trie variable that is the last node whose index matches the prefix or \emph{t} if there is no matching index in \emph{t's} children.

\item \emph{func (t *trie) Lookup(key []byte) (value interface{}, ok bool)}\\
Function of trie type. It looks up for a key in \emph{t} variable. If the key doesn't exist or the interface returned by prefix function is nil, it returns \emph{nil} and \emph{false}, otherwise, it returns the interface and \emph{true}.

\item \emph{func (t *trie) Insert(key []byte, value interface{})}\\
Function of trie type. It inserts a new interface using the key received as parameter. If the key already exists, it updates the interface value.

\item \emph{func (t *trie) Delete(key []byte)}\\
Function of trie type. It deletes an interface by setting it to nil according to the key received as parameter. It searches and records the deepest node with multiple children and siblings.

\end{itemize}