\definecolor{mygreen}{rgb}{0,0.6,0}
\definecolor{mygray}{rgb}{0.5,0.5,0.5}
\definecolor{mymauve}{rgb}{0.58,0,0.82}

\lstset{
	breaklines=true, 
	numbers=left, 
	numberstyle=\tiny\color{mygray}, 
	stringstyle=\color{mymauve}, 
	keywordstyle=\color{blue}, 
	commentstyle=\color{mygreen}
}
\section*{IR Instructions}
All the instructions are defined by this data structure:
\begin{lstlisting}
type Intruction struct{
	op	int
	arg1	Symbol
	arg2	Symbol
	result	Symbol
	jump	*Instruction
}
\end{lstlisting}
All the instructions will be stored in a array, in this way there is no need to have a field of the data structure that points to the next instruction. In case of a jump, conditional or unconditional, the next, possible, instruction is stored in the field \emph{jump}.

\subsection*{Binary Operations}
The binary operations defined in \textbf{EL} are:
\begin{itemize}
\item arithmetical operations (\emph{SUM}, \emph{SUB}, \emph{MUL}, \emph{DIV});
\item logic operations (\emph{AND}, \emph{OR});
\item comparison operations (\emph{EQUAL}, \emph{NOT\_EQUAL}, \emph{LOWER}, \emph{GREATER}, \emph{LOWER\_EQUAL}, \emph{GREATER\_EQUAL}).
\end{itemize}
An example is:\\
\textbf{EL} instruction: \[x=y+w+z\]
intermediate representation: \[Istr_1\mathtab\mathtab op:ADD\mathtab arg1:y\mathtab arg2:w\mathtab result:x_1\mathtab true:NULL\mathtab false:NULL\mathtab next:Istr_2\]
\[Istr_2\mathtab\mathtab op:ADD\mathtab arg1:x_1\mathtab arg2:z\mathtab result:x\mathtab true:NULL\mathtab false:NULL\mathtab next:Istr_3\]
\[\ldots\]  
\subsection*{Unary Operations}

The unary operations defined in \textbf{EL} are:

\begin{itemize}

	\item arithmetic operation (UNARY\_MINUS);
	
	\item logic operation (NOT);
	
	\item operations on addresses (ASSIGNMENT).
	
\end{itemize}

An example is:\\
\textbf{EL} instruction:
\begin{table}[H]
\centering
\begin{tabular}{ll}
\$x=-\$y
\end{tabular}
\end{table}
\tab\\
Intermediate representation:
\begin{table}[H]
\centering
\begin{tabular}{llllll}
$Instr_1$ & op: UNARY\_MINUS & arg1: \$y & arg2: \emph{nil} & result: $\$x_1$ & jump: \emph{nil}\\
$Instr_2$ & op: ASSIGNMENT & arg1: $\$x_1$ & arg2: \emph{nil} & result: $\$x$ & jump: \emph{nil}\\
$Instr_3$ & \ldots
\end{tabular}
\end{table} 

\subsection*{Unconditional Jumps}
An unconditional jump occurs when there is a \emph{jump} that does not depend on the evaluation of any condition. Keywords such as \emph{break} and \emph{continue} are examples of unconditional jumps. Here's an example of \emph{break} keyword:

\textbf{EL} instruction:
\begin{table}[H]
\centering
\begin{tabular}{l}
for \$i = 0; \$i $<$ 10; \$i = \$i + 1 \{\\
\tab break\\
\}
\end{tabular}
\end{table}
Intermediate representation:
\begin{table}[H]
\centering
\begin{tabular}{llllll}
$Instr_1$ & op: ASSIGNMENT & arg1: 0 & arg2: \emph{nil} & result: $\$i$ & jump: \emph{nil}\\
\emph{Label} & \emph{CONDITION}\\
$Instr_2$ & op: GREATER\_EQUAL & arg1: \$i & arg2: 10  & result: $t_1$ & jump: \emph{nil}\\
$Instr_3$ & op: C\_JUMP & arg1: $t_1$ & arg2: \emph{nil} & result: \emph{nil} & jump: \emph{OUT} \\
$Instr_4$ & op: ADD & arg1: \$i & arg2: 1 & result: $\$i_1$ & jump: \emph{nil}\\
$Instr_5$ & op: ASSIGNMENT & arg1: $\$i_1$ & arg2: \emph{nil} & result: $\$i$ & jump: \emph{nil}\\
$Instr_6$ & op: U\_JUMP & arg1: \emph{nil} & arg2: \emph{nil} & result: \emph{nil} & jump: \emph{CONDITION}\\
\emph{Label} & \emph{OUT}\\
$Instr_7$ & \dots\\
\end{tabular}
\end{table}
Here's an example of \emph{continue} keyword:

\textbf{EL} instruction:
\begin{table}[H]
\centering
\begin{tabular}{l}
for \$i = 0; \$i $<$ 10; \$i = \$i + 1 \{\\
\tab continue\\
\}
\end{tabular}
\end{table}
Intermediate representation:
\begin{table}[H]
\centering
\begin{tabular}{llllll}
$Instr_1$ & op: ASSIGNMENT & arg1: 0 & arg2: \emph{nil} & result: $\$i$ & jump: \emph{nil}\\
\emph{Label} & \emph{CONDITION}\\
$Instr_2$ & op: GREATER\_EQUAL & arg1: \$i & arg2: 10  & result: $t_1$ & jump: \emph{nil}\\
$Instr_3$ & op: C\_JUMP & arg1: $t_1$ & arg2: \emph{nil} & result: \emph{nil} & jump: \emph{OUT}\\
$Instr_4$ & op: U\_JUMP & arg1: \emph{nil} & arg2: \emph{nil} & result: \emph{nil} & jump: \emph{CONDITION}\\
$Instr_5$ & op: ADD & arg1: \$i & arg2: 1 & result: $\$i_1$ & jump: \emph{nil}\\
$Instr_6$ & op: ASSIGNMENT & arg1: $\$i_1$ & arg2: \emph{nil} & result: $\$i$ & jump: \emph{nil}\\
$Instr_7$ & op: U\_JUMP & arg1: \emph{nil} & arg2: \emph{nil} & result: \emph{nil} & jump: \emph{CONDITION}\\
\emph{Label} & \emph{OUT}\\
$Instr_8$ & \dots\\
\end{tabular}
\end{table}
\subsection*{Conditional Jumps}
A conditional jump occurs when there is a \emph{jump} that depends on the evaluation of a condition. Control statements like \emph{IF\_THEN}, \emph{IF\_THEN\_ELSE} and \emph{FOR} use conditional jumps. Here's an example of \emph{IF\_THEN\_ELSE} control statement: 

\textbf{EL} code:
\begin{table}[H]
\centering
\begin{tabular}{ll}
if \$a $<$ \$b \{\\
\tab \$x=\$y\\
\}\\
else \{\\
\tab \$x=\$z\\
\}
\end{tabular}
\end{table}

Intermediate representation:
\begin{table}[H]
\centering
\begin{tabular}{llllll}
$Instr_1$ & op: GREATER\_EQUAL & arg1: \$a & arg2: \$b & result: $t_1$ & jump: \emph{nil}\\
$Instr_2$ & op: C\_JUMP & arg1: $t_1$ & arg2: \emph{nil} & result: \emph{nil} & jump: \emph{ELSE}\\
$Instr_3$ & op: ASSIGNMENT & arg1: \$y & arg2: \emph{nil} & result: \$x & jump: \emph{nil}\\
$Instr_4$ & op: U\_JUMP& arg1: \emph{nil} & arg2: \emph{nil} & result: \emph{nil} & jump: \emph{OUT}\\
\emph{Label} & \emph{ELSE}\\
$Instr_5$ & op: ASSIGNMENT & arg1: \$z & arg2: \emph{nil} & result: \$x & jump: \emph{nil}\\
\emph{Label} & \emph{OUT}\\
$Instr_6$ & \dots\\
\end{tabular}
\end{table}

\subsection*{Function Call}

The IR for a function call is as follows:

\begin{table}[H]
\centering
\begin{tabular}{ll}
\$x = some\_function("\%d", \$k+1)
\end{tabular}
\end{table}

becomes:

\begin{table}[H]
\centering
\begin{tabular}{ll}
$Instr_1$ & op: PARAM \tab arg1: "\%d" \tab arg2: NULL \tab result: NULL\tab true: NULL\tab false: NULL\tab next: $Istr_2$\\
$Instr_2$ & op: ADD \tab arg1: \$k \tab arg2: 1 \tab result: $t_1$ \tab true: \emph{TRUE}\tab false: \emph{FALSE}\tab next: $Instr_3$\\
$Instr_3$ & op: PARAM \tab arg1: $t_1$ \tab arg2: NULL \tab result: NULL \tab true: NULL \tab false: NULL \tab next: $Instr_4$\\
$Instr_4$ & op: CALL \tab arg1: some\_function \tab arg2: 2 \tab result: $t_2$ \tab true: NULL \tab false: NULL \tab next: $Instr_5$\\
$Instr_5$ & op: ASSIGNMENT \tab arg1: $t_2$ \tab arg2: NULL \tab result: \$x \tab true: NULL \tab false: NULL \tab next: $Instr_6$\\
$Instr_6$ & \dots\\
\end{tabular}
\end{table}

So, first of all the function parameters are evaluated and then the function call is performed.

Since we can have nested function calls, it is necessary to keep track of the number of parameters of each function; we do that using in the CALL instruction the number of needed parameters as second argument (the first one is the called function).

The run-time routines will handle procedure parameter passing, calls and return operations. The CALL instruction will execute the arg1 function using the arg2 needed parameters.

