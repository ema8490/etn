\subsection*{Conditional Jumps}
A conditional jump occurs when there is a \emph{jump} that depends on the evaluation of a condition. Control statements like \emph{IF\_THEN}, \emph{IF\_THEN\_ELSE} and \emph{FOR} use conditional jumps. Here's an example of \emph{IF\_THEN\_ELSE} control statement: 

\textbf{EL} code:
\begin{table}[H]
\centering
\begin{tabular}{ll}
if \$a $<$ \$b \{\\
\tab \$x=\$y\\
\}\\
else \{\\
\tab \$x=\$z\\
\}
\end{tabular}
\end{table}

Intermediate representation:
\begin{table}[H]
\centering
\begin{tabular}{llllll}
$Instr_1$ & op: GREATER\_EQUAL & arg1: \$a & arg2: \$b & result: $t_1$ & jump: \emph{nil}\\
$Instr_2$ & op: C\_JUMP & arg1: $t_1$ & arg2: \emph{nil} & result: \emph{nil} & jump: \emph{ELSE}\\
$Instr_3$ & op: ASSIGNMENT & arg1: \$y & arg2: \emph{nil} & result: \$x & jump: \emph{nil}\\
$Instr_4$ & op: U\_JUMP& arg1: \emph{nil} & arg2: \emph{nil} & result: \emph{nil} & jump: \emph{OUT}\\
\emph{Label} & \emph{ELSE}\\
$Instr_5$ & op: ASSIGNMENT & arg1: \$z & arg2: \emph{nil} & result: \$x & jump: \emph{nil}\\
\emph{Label} & \emph{OUT}\\
$Instr_6$ & \dots\\
\end{tabular}
\end{table}
