\subsection*{Conditional Jumps}
A conditional jump occurs when there is a \emph{jump} that depends on the evaluation of a condition. Control statements like \emph{IF\_THEN}, \emph{IF\_THEN\_ELSE} and \emph{FOR} use conditional jumps. Here's an example of \emph{IF\_THEN\_ELSE} control statement: 

\textbf{EL} code:
\begin{table}[H]
\centering
\begin{tabular}{ll}
if \$a $<$ \$b \{\\
\tab \$x=\$y\\
\}\\
else \{\\
\tab \$x=\$z\\
\}
\end{tabular}
\end{table}

Intermediate representation:
\begin{table}[H]
\centering
\begin{tabular}{ll}
$Instr_1$ & op: L\_COMPARISON\tab arg1: \$a \tab arg2: \$b \tab result: $t_1$\tab true: NULL\tab false: NULL\tab next: $Istr_2$\\
$Instr_2$ & op: BNEQ \tab arg1: $t_1$ \tab arg2: NULL result: NULL \tab true: \emph{TRUE}\tab false: \emph{FALSE}\tab next: NULL\\
\emph{Label} & \emph{TRUE}\\
$Instr_3$ & op: ASSIGNMENT\tab arg1: \$y \tab result: \$x \tab true: NULL \emph{BODY}\tab false: NULL \tab next: \emph{OUT}\\
\emph{Label} & \emph{FALSE}\\
$Instr_4$ & op: ASSIGNMENT\tab arg1: \$z \tab result: \$x \tab true: NULL \emph{BODY}\tab false: NULL \tab next: \emph{OUT}\\
\emph{Label} & \emph{OUT}\\
$Instr_5$ & \dots\\
\end{tabular}
\end{table}
