\subsection*{Function Call}

The IR for a function call is as follows:

\begin{table}[H]
\centering
\begin{tabular}{ll}
\$x = some\_function("\%d", \$k+1)
\end{tabular}
\end{table}

becomes:

\begin{table}[H]
\centering
\begin{tabular}{ll}
$Instr_1$ & op: PARAM \tab arg1: "\%d" \tab arg2: NULL \tab result: NULL\tab true: NULL\tab false: NULL\tab next: $Istr_2$\\
$Instr_2$ & op: ADD \tab arg1: \$k \tab arg2: 1 \tab result: $t_1$ \tab true: \emph{TRUE}\tab false: \emph{FALSE}\tab next: $Instr_3$\\
$Instr_3$ & op: PARAM \tab arg1: $t_1$ \tab arg2: NULL \tab result: NULL \tab true: NULL \tab false: NULL \tab next: $Instr_4$\\
$Instr_4$ & op: CALL \tab arg1: some\_function \tab arg2: 2 \tab result: $t_2$ \tab true: NULL \tab false: NULL \tab next: $Instr_5$\\
$Instr_5$ & op: ASSIGNMENT \tab arg1: $t_2$ \tab arg2: NULL \tab result: \$x \tab true: NULL \tab false: NULL \tab next: $Instr_6$\\
$Instr_6$ & \dots\\
\end{tabular}
\end{table}

So, first of all the function parameters are evaluated and then the function call is performed.

Since we can have nested function calls, it is necessary to keep track of the number of parameters of each function; we do that using in the CALL instruction the number of needed parameters as second argument (the first one is the called function).

The run-time routines will handle procedure parameter passing, calls and return operations. The CALL instruction will execute the arg1 function using the arg2 needed parameters.
