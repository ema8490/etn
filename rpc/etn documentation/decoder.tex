\definecolor{mygreen}{rgb}{0,0.6,0}
\definecolor{mygray}{rgb}{0.5,0.5,0.5}
\definecolor{mymauve}{rgb}{0.58,0,0.82}

\lstset{
	language=C, 
	breaklines=true, 
	numbers=left, 
	numberstyle=\tiny\color{mygray}, 
	stringstyle=\color{mymauve}, 
	keywordstyle=\color{blue}, 
	commentstyle=\color{mygreen}
}

\section*{Decoder}
\emph{EtnDecoder} is a \lq\lq reader" and can be considered a base class. This data structure contains a function pointer to a \emph{read} function, a pointer to the data, the data length and an index to the next data.
\begin{lstlisting}
typedef struct EtnDecoder_s {
	int (*read) (struct EtnDecoder_s *d, uint8_t *data, EtnLength length);
	void **indexToData;
	EtnLength indexToDataLength;
	EtnLength nextIndex;
} EtnDecoder;
\end{lstlisting}
The actual implementation of \emph{read} is defined in file \emph{decoder.c}.\\
\emph{EtnBufferDecoder} is a specialization of \emph{EtnDecoder}. It decodes to a provided memory range. In contains a \emph{EtnDecoder} and two pointers to the memory range.
\begin{lstlisting}
typedef struct EtnBufferDecoder_s {
	EtnDecoder decoder;
	uint8_t *dataCurrent;
	uint8_t *dataEnd;
} EtnBufferDecoder;
\end{lstlisting}
\emph{PathDecoder} is a specialization of \emph{EtnDecoder}.
It decodes from a provided path. It contains a \emph{EtnBufferDecoder} and a pointer to the original data.
\begin{lstlisting}
typedef struct EtnPathDecoder_s {
	EtnBufferDecoder bufferDecoder;
	uint8_t *dataOriginal;
} EtnPathDecoder;
\end{lstlisting}
