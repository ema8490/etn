\definecolor{mygreen}{rgb}{0,0.6,0}
\definecolor{mygray}{rgb}{0.5,0.5,0.5}
\definecolor{mymauve}{rgb}{0.58,0,0.82}

\lstset{
	language=C, 
	breaklines=true, 
	numbers=left, 
	numberstyle=\tiny\color{mygray}, 
	stringstyle=\color{mymauve}, 
	keywordstyle=\color{blue}, 
	commentstyle=\color{mygreen}
}

\section*{Types}
\textbf{ETN} types are all defined in file \emph{types.h}.
As we can see, the encoders/decoders do not interact directly with the types. Indeed, the encoder/decoder type does not contain the type to be encoded/decoded. The interaction between them comes from other data structures that contain them all.\\
For instance, data structure \emph{EtnRpcHost} (defined in \emph{erpc.c} file and used for interprocess-communication):
\begin{lstlisting}
struct EtnRpcHost {
	EtnValue v;
	EtnEncoder *e;
	EtnDecoder *d;
};
\end{lstlisting}
Here an example (from file \emph{testPointer.c}) on how a value of type \emph{pointer} is encoded and decoded (actions needed to encode/decode other types are similar).


\begin{lstlisting}
static bool _predicate(void){
	uint8_t *in = malloc (sizeof (uint8_t)), *out;
	*in = 0xbe;

	uint8_t buf[1024];
	EtnBufferEncoder *e = etnBufferEncoderNew(buf, sizeof(buf));
	EtnLength encodedSize;
	etnEncode((EtnEncoder *) e, EtnToValue(&Uint8PtrType, &in), &encodedSize);

	EtnBufferDecoder *d = etnBufferDecoderNew(buf, encodedSize);
	etnDecode((EtnDecoder *) d, EtnToValue(&Uint8PtrType, &out));

	free (e);
	etnBufferDecoderFree (d);

	return *in == *out;
}
\end{lstlisting}
A pointer to a \emph{uint8\_t} is allocated and initialized at the value \emph{0xbe} (\emph{in}), while another one is declared (\emph{out}). An \emph{EtnBufferEncoder} is created and the function \emph{etnEncode()} is called to encode the value of the \emph{in} pointer, which is going to be written into the buffer. Then, an \emph{EtnBufferDecoder} is created using the same buffer of \emph{EtnBufferEncoder}. The function \emph{etnDecode()} is used to decode the content of the buffer and the result is inside \emph{out} pointer. Eventually, the test returns true if the content of the two pointers (\emph{in} and \emph{out}) is the same, false otherwise.\\
So, we can notice the only link between the \emph{EtnEncoder} and \emph{EtnDecoder} is the common buffer \emph{buf}. The \emph{macro} \emph{EtnToValue} takes in input the \emph{type} and the value and returns a variable of type \emph{EtnValue} that contains both the pointer to the value (\emph{void*}) and the type (\emph{EtnType}).