\section{rpc.c}

The global variable declared and defined here is:

\begin{itemize}
\item DebugFlagDesc debugFlagDesc[]\\
This is simply an array of debug flags. It is in \emph{rpc.c} because it is used from both \emph{Ethos} and \emph{Dom0} even though it is not RPC-specific. 
\end{itemize}
In this file, there are some debug purpose functions. They are used to send a ping between both Shadowdaemon and Terminal. Here they are:

\begin{itemize}

\item \emph{void rpcShadowDaemonPing (EtnRpcHost *h, uint64\_t eventId)}

\item \emph{void rpcShadowDaemonPingReply (EtnRpcHost *h, uint64\_t eventId, Status status)}

\item \emph{void rpcTerminalPing (EtnRpcHost *h, uint64\_t eventId)}

\item \emph{void rpcTerminalPingReply (EtnRpcHost *h, uint64\_t eventId, Status status)}

\end{itemize}
The other functions defined here are:

\begin{itemize}

\item \emph{void rpcInitInterfaces(void)}\\
This function initialize the interfaces. All the variables that are initialized are global variables declared in \emph{rpc.h} and defined in \emph{packetEncoder.c}. For each Ethos primary component a packet encoder is defined. Then, a null encoder and a buffer decoder are also defined. These five interfaces, in addiction to the servers of each primary component, are used to define the interface host of each primary component.

\item \emph{void rpcInit(void)}\\
This function calls some other functions that initialize rpc components. The declaration is inside \emph{rpc.h}.
\end{itemize}