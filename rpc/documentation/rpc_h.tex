\section{rpc.h}

The global variables declared here are:
\begin{itemize}
\item extern PacketEncoder *rpcInterfaceShadowDaemonPacketEncoder
\item extern PacketEncoder *rpcInterfaceTerminalPacketEncoder
\item extern PacketEncoder *rpcInterfaceKernelPacketEncoder
\item extern EtnNullEncoder *rpcInterfaceNullEncoder
\item extern EtnBufferDecoder *rpcInterfaceBufferDecoder
\item extern EtnRpcHost *rpcInterfaceShadowDaemonHost
\item extern EtnRpcHost *rpcInterfaceTerminalHost
\item extern EtnRpcHost *rpcInterfaceKernelHost
\item extern EtnRpcHost *rpcInterfaceNullHost
\end{itemize}
Using extern keyword, C variables are declared but not defined. So, they must be defined in another header or c file before using them. In particular, these variables are defined in \emph{packetEncoder.c}
Moreover, the extern extends the visibility to the whole program. These global variables are used as hosts, encoders and decoders of Ethos primary components.\\\\
The functions exported by the header file are:

\begin{itemize}
\item \emph{void rpcInit(void)}\\
This function calls some other functions that initialize rpc components. The definition is inside \emph{rpc.c}.

\item \emph{rpcCall(fn, host, connection, eventId, ...)}\\
This macro is used to: 
\begin{enumerate}

\item calculate length of encoded packet with a dummy run
\item reset the packet encoder with this length
\item perform the actual RPC call

\end{enumerate}

\end{itemize}